\documentclass[8pt]{beamer}
\usepackage[T2A]{fontenc}                
\usepackage[utf8]{inputenc}  
\usepackage[english, russian]{babel}
\usepackage{indentfirst}
\usepackage{graphicx}
 
\usetheme{Warsaw}
\usecolortheme{seahorse}

\title{Практикум на ЭВМ.\\ Отчет по четвертому заданию.}
\author{ Эмиль Гусейнов, 312 группа }
\institute{МГУ имени М.В. Ломоносова, Москва, Россия}
\date{2017}
 
\begin{document}
 
\frame{\titlepage}
 
	\frame{\frametitle{Постановка задачи}
		\begin{itemize}
			\item Есть два поставщика стали: {\bf Westeros Inc. и Harpy \& Co.} \\*Необходимо выбрать компанию, с которой следует заключить эксклюзивный договор на поставку стали.
			\item Необходимо провести {\bf разведывательный анализ данных} с целью ответа на вопрос: {\it "<С каким из поставщиков стали следует заключить договор?">}
		\end{itemize}\tableofcontents}
			
	\begin{frame}
		\frametitle{Исходные данные}
		\begin{itemize}
			\item Дан CSV-файл с данными о производстве оружия и количестве единиц сломанного оружия за каждый месяц каждым из кузнецов.
		\end{itemize}
	\end{frame} 

\begin{frame}
\frametitle{Цель работы}
Необходимо провести разведывательный анализ данных с целью ответа на вопрос: "С каким из поставщиков стали следует заключить договор?"
\end{frame}

\begin{frame}
\frametitle{Анализ и выводы\\{\smallСредний процент дефектов по месяцам}}
	Чтобы узнать, сколько мечей для следующего месяца нам необходимо произвести, чтобы вооружить наших воинов, мы должны оценить среднее количество сломанных мечей в месяц.
	
	Доля поломанных мечей в месяц n определяется как
	$$ M = \frac{defect_{n}}{\sum_{i=1}^{n-1}produced_{i}-defect_{i}} $$,
	где $ produced_{i} $ и $ defect_{i} $ - количество произведенных и сломанных мечей в месяц $ i $ соответственно.
	
	\centerline{\includegraphics[height=25mm, width=50mm]{failed.png}}
	
Из графика видно, что первые 3 месяца мечи, изготовленные из стали Harpy \& Co практически не ломаются, однако, затем происходит скачок и к 7-му месяцу мы получаем уже 8,3\% дефектных мечей. Что касается Westeros Inc., то она стабильно показывает цифру в 8,5\%

	Итак, эта метрика не дает нам однозначно понять, какая из компаний лучше.

\end{frame}

\begin{frame}
\frametitle{Математическое ожидание жизни меча}
\begin{figure}[h]
		\includegraphics[height=30mm, width=50mm]{expected_value.png}
\end{figure}

Было бы неплохо узнать ``среднюю продолжительность жизни``  меча. В этом нам поможет плотность распределения
	$$ P(X = i) = p_{i} $$ - вероятность того, что меч прослужит $ i $ месяцев.
	\par
	Заметим, что количество месяцев в исходных слишком мало, поэтому мы не получим адекватную величину математического ожидания, однако если предположить, что при $ i > 6 $  $ p_{i} $ для обеих компаний равны, то, сравнив полученные величины, можно сказать с какой компанией следует заключить контракт.
	\par
	Для Westeros Inc. получилось - 27.34, для Harpy \& Co - 27.28. Мы снова не получили результата, который бы дал возможность принять решение.
\end{frame}

\begin{frame}
\frametitle{Количество мечей для похода}

	Попробуем посмотреть на задачу с точки зрения полководца. Пусть у нас есть $ V $ воинов, и мы оцениваем продолжительность военной кампании в $ S $ месяцев. В течение $ S $ месяцев каждый из солдат должен иметь хотя бы по одному мечу. 
	\par
	Если мы знаем какой процент мечей ломается каждый месяц, то можно сказать сколько мечей необходимо изготовить. 
	Мы выяснили, что для Westeros Inc. процент поломок примерно постоянен и равен 8.5 \%. Для Harpy \& Co процент поломок в последние 3 месяца 5.8\%, 7.3\%, 8.3\%. Предположим, что далее он постоянен и равен 9\%.
	Количество мечей, которое нам необходимо произвести равно
		
		$$ M = \frac{V}{\prod_{i=1}^S(1 - p_{i})} $$
\end{frame}


\begin{frame}
\frametitle{Количество мечей для похода}
\begin{figure}[h]
	\includegraphics[height=30mm, width=50mm]{swords.png}
\end{figure}
	Из графика, видно, что для кампаний длиной в 2 и менее года выгоднее сотрудничать с Harpy \& Co. Расчеты показывают, что на дистанции в 4 года она все еще предпочтительнее своего конкурента.
\end{frame}

\begin{frame}
\frametitle{Количество мечей для содержания гарнизона}
\begin{figure}[h]
	\includegraphics[height=30mm, width=50mm]{troops.png}
\end{figure}
	Теперь предположим, что мы тренируем бойцов для будущего похода, и нам необходимо снабдить каждого бойца своим мечем. 
	Пусть у нас $ V $ солдат. Тогда в первый месяц сломается $ p_{1} * V $ мечей, после чего мы закупимся до $ V $ мечей, на второй месяц мы купим $ p_{2} * V $, за $ S $ же месяцев $$ M = V * \sum\limits_{i=1}^S p_{i} $$
	Мы получили схожий результат с тем, что имели в предыдущей задаче.
\end{frame}

\begin{frame}
\frametitle{Заключение}
	Итак, мы рассмотрели две возможные проблемы, с которыми могут столкнуться полководцы королевы. В каждой из задач компания Harpy \& Co превосходит своего конкурента на дистанции в 4+ года, поэтому я рекомендую подписать контракт именно с ней. 
\end{frame}

\end{document}