\documentclass[8pt]{beamer}
\usepackage[T2A]{fontenc}                
\usepackage[utf8]{inputenc}  
\usepackage[english, russian]{babel}
\usepackage{indentfirst}
\usepackage{graphicx}

\title{Практикум на ЭВМ.\\ Отчет по четвертому заданию.}
\author{Гриценко Богдан, 311 группа.\\ Гусейнов Эмиль, 312 группа.}
\institute{МГУ имени М.В. Ломоносова, Москва, Россия}
\date{2017}
 
\begin{document}
 
\frame{\titlepage}
 
\begin{frame}
\frametitle{Постановка задачи}
После оглушительного успеха в освобождении Астапора, Миэрина и Юнкая от власти работорговцев Дейенерис Бурерожденная открыла себе доступ к Летнему морю, а следовательно -- путь в Вестерос.\\
Для ведения войны с Семью Королевствами нужно оружие, а для оружия нужна сталь. Нет никаких сомнений в кузнечном искусстве Безупречных, однако поставщики стали не столь надежны.\\
Два основных поставщика стали -- это {Westeros Inc.} и {Harpy \& Co}. На протяжении нескольких месяцев мы закупаем сталь у обеих компаний, и каждая из них предлагает ощутимую скидку при заключении эксклюзивного договора на поставку.\\
\end{frame}

\begin{frame}
\frametitle{Постановка задачи}
Советник королевы Тирион Ланнистер знает о твоем умении принимать взвешенные рациональные решения и просит помощи в объективном решении вопроса о том, с какой из компаний следует заключить эксклюзивный договор на поставку стали.\\
У Тириона есть записи о производстве мечей каждым из кузнецов-безупречных, а также данные о количестве сломанных мечей в каждый из месяцев ведения боевых действий.\\
\end{frame}
 
\begin{frame}
\frametitle{Цель работы}
Необходимо провести разведывательный анализ данных с целью ответа на вопрос: "С каким из поставщиков стали следует заключить договор?"
\end{frame}

\begin{frame}
\frametitle{Анализ\\{\smallСредний процент дефектов в зависимости от возраста меча}}
\begin{figure}[h]
		\includegraphics[width=90mm]{0.png}
		\caption{1}
		\label{First}
\end{figure}
harpy.co - они показывают отличную надёжность в первые 3 месяца, а далее - часто ломаются.

westeros.inc - они же показывают стабильный процент поломок - практически константа.
\end{frame}

\begin{frame}
\frametitle{Анализ \\{\small Средние объемы поставок}}
\begin{figure}[h]
		\includegraphics[width=90mm]{1.png}
		\caption{2}
		\label{Second}
\end{figure}

Оказывается, что средние объёмы поставок у обеих компаний практически одинаковы, независимо от месяца. Что не даёт полноценно исследовать гипотезу об ухудшении качества в зависимости от количества. Зато говорит о стабильных поставках обеих фирм.
\end{frame}

\begin{frame}
\frametitle{Анализ\\ {\small Отношение общего числа дефектов к общему числу произведенной продукции (Рис. 3) и производная этого отношения (Рис. 4)}}
\begin{figure}[h]
		\includegraphics[width=70mm]{2.png}
		\caption{3}
		\label{Third}
\end{figure}

\begin{figure}[h]
		\includegraphics[width=70mm]{2b.png}
		\caption{4}
		\label{Fourth}
\end{figure}

\end{frame}

\begin{frame}
\frametitle{Анализ\\ {\small Отношение общего числа дефектов к общему числу произведенной продукции (Рис. 3) и производная этого отношения (Рис. 4)}}
Данные графики могли бы хорошо помочь при анализе гипотезы во втором пункте. Также по ним можно судить об изменении качества в целом, однако для компенсации погрешности необходимы данные для большего временного промежутка. В общем, процент поломок растет (Рис. 3) и рост увеличивается, (Рис. 4) но это можно списать на то, что в первые месяцы все мечи новые.
\end{frame}


\begin{frame}
\frametitle{Выводы}
harpy.co подходит для коротких войн, т.к. очень стабильны в первые 3 месяца, \\далее растут риски.\\
~\\
westeros.inc очень стабильно выходят из строя. Их следует заказывать для длительных походов с заранее расчитанным запасом.
\end{frame}

\begin{frame}{Задание выполняли}
	\begin{itemize}
		{\small
		\item Гриценко Богдан, студент 311 группы.
		\item Гусейнов Эмиль, студент 312 группы.}
	\end{itemize}	

\end{frame}

\end{document}
