\documentclass[8pt]{beamer}
\usepackage[T2A]{fontenc}                
\usepackage[utf8]{inputenc}  
\usepackage[english, russian]{babel}
\usepackage{indentfirst}
\usepackage{graphicx}

\title{Практикум на ЭВМ.\\ Отчет по четвертому заданию.}
\author{Гриценко Богдан, 311 группа.}
\institute{МГУ имени М.В. Ломоносова, Москва, Россия}
\date{2017}
 
\begin{document}
 
\frame{\titlepage}
 
\begin{frame}
\frametitle{Постановка задачи}
После оглушительного успеха в освобождении Астапора, Миэрина и Юнкая от власти работорговцев Дейенерис Бурерожденная открыла себе доступ к Летнему морю, а следовательно -- путь в Вестерос.\\
Для ведения войны с Семью Королевствами нужно оружие, а для оружия нужна сталь. Нет никаких сомнений в кузнечном искусстве Безупречных, однако поставщики стали не столь надежны.\\
Два основных поставщика стали -- это {Westeros Inc.} и {Harpy \& Co}. На протяжении нескольких месяцев мы закупаем сталь у обеих компаний, и каждая из них предлагает ощутимую скидку при заключении эксклюзивного договора на поставку.\\
\end{frame}

\begin{frame}
\frametitle{Постановка задачи}
Советник королевы Тирион Ланнистер знает о твоем умении принимать взвешенные рациональные решения и просит помощи в объективном решении вопроса о том, с какой из компаний следует заключить эксклюзивный договор на поставку стали.\\
У Тириона есть записи о производстве мечей каждым из кузнецов-безупречных, а также данные о количестве сломанных мечей в каждый из месяцев ведения боевых действий.\\
\end{frame}
 
\begin{frame}
\frametitle{Цель работы}
Необходимо провести разведывательный анализ данных с целью ответа на вопрос: "С каким из поставщиков стали следует заключить договор?"
\end{frame}

\begin{frame}
\frametitle{Метрики\\}
1а) Стабильность объёма поставок стали.\\
1б) Стабильность качества стали по месяцам. В качестве критерия качества выступает вероятность сломаться в течение следующего месяца\\
Для этих критериев можно было бы выработать чёткую метрику, типа дисперсии, но мы просто посмотрим на графики, тем более что самая важный критерий - следующий\\
2) Количество целых мечей - чем больше мечей, тем больше войско. Ни количество произведённых мечей, ни количество сломанных сами по себе не важны так, как их разность. Мы попытаемся предсказать сколько мечей у нас будет в будущем - вплоть до двух лет.\\
\end{frame}

\begin{frame}
\frametitle{Анализ \\{\small Объемы поставок}}
\begin{figure}[h]
		\includegraphics[width=70mm]{prod.png}
\end{figure}

Обе компании очень стабильно поставляют сталь для мечей. Тут всё ясно.
\end{frame}

\begin{frame}
\frametitle{Анализ\\ {\small Шансы сломаться в течение следующего месяца в зависимости от возраста меча и месяца изготовления}}
\begin{figure}[h]
		\includegraphics[width=45mm]{hdef.png}
\end{figure}

\begin{figure}[h]
		\includegraphics[width=45mm]{wdef.png}
\end{figure}

\end{frame}

\begin{frame}
\frametitle{Анализ\\ {\small Шансы сломаться в течение следующего месяца в зависимости от возраста меча и месяца изготовления}}
По графикам видно, что шансы сломаться почти не зависят от месяца изготовления: у westeros.inc разброс побольше, но всё равно невелик. Обе компании поставляют сталь примерно одинакового качества от месяца к месяцу. Таким образом, по первому критерию обе компании нас устраивают: можно не бояться форс-мажоров.\\
А теперь самое интересное. Зависимость от возраста.\\
Мечи из стали westeros.inc ломаются практически одинаково, независимо от своего возраста. Естественно предположить, что в будущем эта тенденция сохранится. Совсем другое дело - harpy.co. Её мечи ломаются практически с константной частотой первые три месяца. Потом частота поломок резко увеличивается и переходит на новое "константное" значение. Тут уже сложнее что либо гарантировать, однако логичнее всего мне видется предположить, что шансы сломать меч останутся на этом новом значении.
\end{frame}

\begin{frame}
\frametitle{Предсказание\\ {\small Количество оставшихся целых мечей через некоторое время}}
Я выстроил своё предсказание следующим образом:\\
Так как поставки стабильны я положил для обоих компаний для k > 6 поставки на k-ый месяц равными среднему значению поставок в первые 6 месяцев.\\
Так как качество стали стабильно я положил для k <= 6 шансы сломаться в течение месяца в k-ый месяц после изготовления равными среднему значению шансов сломаться для меча той же компании и такого же возраста.\\
Для k > 6 шансы сломатся в течение месяца я высчитал в соответствии с предположениями из предыдущего слайда. Для westeros.inc это средние шансы по всем шести месяцам, для harpy.co это среднее по последним трём месяцам.
\end{frame}

\begin{frame}
\frametitle{Предсказание\\ {\small Количество оставшихся целых мечей через некоторое время}}

\begin{figure}[h]
		\includegraphics[width=80mm]{alive.png}
\end{figure}
Спустя 6 месяцев - до этого момента график построен без каких-либо предположений - harpy.co выглядит предпочтительнее. Но потом происходит интересное. Спустя 11 месяцев объёмы совпадают, а спустя 24 westeros.inc уже лидирует со значительным запасом.
\end{frame}

\begin{frame}
\frametitle{Выводы}
Так как захват целого континента - дело небыстрое, я рекомендую westeros.inc
\end{frame}

\begin{frame}{Задание выполнял}
	\begin{itemize}
		{\small
		\item Гриценко Богдан, студент 311 группы.}
	\end{itemize}	

\end{frame}


\end{document}
