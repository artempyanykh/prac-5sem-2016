\documentclass[10pt,pdf,hyperref={unicode}]{beamer}
\usepackage{lmodern} 
\usepackage[T2A]{fontenc}
\usepackage[utf8]{inputenc}
\setbeamertemplate{navigation symbols}{}
\usetheme{Warsaw}
\setbeamercolor{structure}{fg=cyan!90!black}




\title{Результаты анализа и выводы}   
\author{Задание выполнили: Даруш Н., Кенесова А., Яралиева Ф.} 
\institute{Московский Государственный Университет М.В Ломоносова\\ Факультет Вычислительной математики и Кибернетики\\ Кафедра Исследования Операций}

\date{Москва, 2017} 

\begin{document}
	\begin{frame}
		\titlepage
	\end{frame} 

	\begin{frame}
	\frametitle{Постановка задачи} 
Имеются 2 поставщика стали это Westeros Inc. и Harpy $ \& $ Co. Для каждой компании есть данные в формате CSV о производстве оружия и количестве единиц сломанного оружия за каждый месяц. Используя эти данные, нам необходимо провести разведывательный анализ и ответить на вопрос: "С каким поставщиком стали следует заключить договор?". 
	\end{frame}

	\begin{frame}
	\frametitle{Описание решения задачи}
	\begin{itemize}
		\item Использовали три критерия оценки работы каждой из компаний:\\ 
		\begin{enumerate}
			\item Общий объем производства оружия\\
			\item Распредления числа поломок по сроку использования для каждого месяца производства в отдельности \\
			\item Распредления числа поломок по сроку использования после каждого месяца эксплуатации, независимо от месяца производства мечей
		\end{enumerate}
	\end{itemize}
	\end{frame}

	\begin{frame}
	\frametitle{Первый критерий} 
По графику видно, что общие объемы производства мечей каждой из компаний совпадают. Отсюда следует, что мы не можем сделать каких-то конкретных выводов о том, какая из компаний предпочтительнее.
	\center{\includegraphics[scale=0.5]{hw.png}\\ сравнение объемов производства двух компаний}
	\end{frame}
	
	\begin{frame}
	\frametitle{Второй критерий} 
Эти графики показывают распределение поломок мечей, произведенных в первый месяц для двух компаний. Мы видим, что в первые три месяца поломки мечей у компании Harpy $\&$ Co минимальны, в отличии от Westeros Inc, где с первого месяца имеется большое число поломок.Но зато у компании Harpy $\&$ Co начиная с 4-го месяца резко увеличивается число поломок, тогда как у Westeros Inc распределение поломок почти равномерно, т.е каждый месяц ломается почти одно и тоже количество мечей. 
	\center{\includegraphics[scale=0.5]{1m.png}\\}
	\end{frame}

	\begin{frame}
	\frametitle{Второй критерий} 
Таким же образом, рассмотрим дальнейшую судьбу мечей, произведенных в следующие месяца. 
		\includegraphics[scale=0.4]{2m.png}
		\includegraphics[scale=0.4]{3m.png}\\
		\center{\includegraphics[scale=0.3]{4m.png}\\}
	\end{frame}

	\begin{frame}
	\frametitle{Второй критерий} 
		\includegraphics[scale=0.4]{5m.png}
		\includegraphics[scale=0.4]{6m.png}\\
	\end{frame}

	\begin{frame}
	\frametitle{Вывод и прогноз по второму критерию}
		\center{По графикам видно, что вывод, сделанный нами  для поломок мечей обеих компаний, произведенных в первый месяц, характерен и для поломок мечей, произведенных в остальные. В итоге, исходя из графиков компании Harpy $\&$ Co мы видим, что с 5 месяца число поломок уменьшается, тогда  можно сказать, что в последующие месяца число поломок также будет уменьшаться, либо достигнет некоторой асимптоты. У Westeros Inc. распределение поломок на протяжении всех месяцев равномерно. Тогда по данным хотя бы за первые три месяца можно сделать вывод, что Harpy $\&$ Co предпочтительнее.}

	\end{frame}

	\begin{frame}
	\frametitle{Третий критерий }
График показывает число поломок мечей после каждого месяца их эксплуатации, независимо от того, в каком месяце они были произведены. 
		\center{\includegraphics[scale=0.5]{all.png}\\}
Мы видим, что графики подтверждают выводы, сделанные ранее.
	\end{frame}

	\begin{frame}
	\frametitle{Результаты работы}
		\center{Harpy $\&$ Co лучше, в независимости от продолжительности эксплуатации оружия, как и от продолжительности войны. Таким образом, нам стоит заключить договор с компанией Harpy $\&$ Co. \\}
	\end{frame}

\end{document}