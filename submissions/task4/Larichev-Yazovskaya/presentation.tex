\documentclass[8pt]{beamer}
\usepackage[T2A]{fontenc}                
\usepackage[utf8]{inputenc}  
\usepackage[english, russian]{babel}
\usepackage{indentfirst}
\usepackage{graphicx}
 
\usetheme{Warsaw}
\usecolortheme{seahorse}

\title{Практикум на ЭВМ.\\ Отчет по четвертому заданию.}
\author{ Никита Ларичев, 311 группа.\\ Наталья Язовская, 312 группа.}
\institute{МГУ имени М.В. Ломоносова, Москва, Россия}
\date{2017}
 
\begin{document}
 
\frame{\titlepage}
 
\begin{frame}
\frametitle{Постановка задачи}
После оглушительного успеха в освобождении Астапора, Миэрина и Юнкая от власти работорговцев Дейенерис Бурерожденная открыла себе доступ к Летнему морю, а следовательно -- путь в Вестерос.\\
Для ведения войны с Семью Королевствами нужно оружие, а для оружия нужна сталь. Нет никаких сомнений в кузнечном искусстве Безупречных, однако поставщики стали не столь надежны.\\
Два основных поставщика стали -- это {Westeros Inc.} и {Harpy \& Co}. На протяжении нескольких месяцев мы закупаем сталь у обеих компаний, и каждая из них предлагает ощутимую скидку при заключении эксклюзивного договора на поставку.\\
\end{frame}

\begin{frame}
\frametitle{Постановка задачи}
Советник королевы Тирион Ланнистер знает о твоем умении принимать взвешенные рациональные решения и просит помощи в объективном решении вопроса о том, с какой из компаний следует заключить эксклюзивный договор на поставку стали.\\
У Тириона есть записи о производстве мечей каждым из кузнецов-безупречных, а также данные о количестве сломанных мечей в каждый из месяцев ведения боевых действий.\\
\end{frame}
 
\begin{frame}
\frametitle{Цель работы}
Необходимо провести разведывательный анализ данных с целью ответа на вопрос: "С каким из поставщиков стали следует заключить договор?"
\end{frame}

\begin{frame}
\frametitle{Анализ и выводы\\{\smallСредний процент дефектов в зависимости от "старости" меча}}
\begin{figure}[h]
		\includegraphics[width=70mm]{1.png}
		\caption{1}
		\label{First}
\end{figure}
harpy.co - подходит для коротких войн, т.к. демонстрируют отличную надёжность в первые 3 месяца, а далее - часто ломаются
westeros.inc - для длинных войн, т.к. процент поломок - практически константа. Поэтому достаточно удобно поддерживать общее количество целых мечей армии, не перегружая кузнецов в различные месяцы.
\end{frame}

\begin{frame}
\frametitle{Анализ и выводы \\{\small Средние объемы поставок}}
\begin{figure}[h]
		\includegraphics[width=70mm]{2.png}
		\caption{2}
		\label{Second}
\end{figure}

Как оказалось, средние объёмы поставок у обеих компаний практически одинаковы, даже независимо от месяца. Что не даёт полноценно исследовать теорию об ухудшении качества в зависимости от количества. Зато говорит о стабильных поставках обеих фирм.
\end{frame}

\begin{frame}
\frametitle{Анализ и выводы\\ {\small Отношение общего числа дефектов к общему числу произведенной продукции (Рис. 3) и производная этого отношения (Рис. 4)}}
\begin{figure}[h]
		\includegraphics[width=60mm]{3.png}
		\caption{3}
		\label{Third}
\end{figure}

\begin{figure}[h]
		\includegraphics[width=60mm]{4.png}
		\caption{4}
		\label{Fourth}
\end{figure}

\end{frame}

\begin{frame}
\frametitle{Анализ и выводы\\ {\small Отношение общего числа дефектов к общему числу произведенной продукции (Рис. 3) и производная этого отношения (Рис. 4)}}
Данные графики могли бы хорошо помочь при исследовании теории в пункте 2. Также по ним можно судить об ухудшении/улучшении качества в целом, но, к сожалению, для компенсации погрешности необходимо покрытие большего временного промежутка. В целом, процент поломок растет (Рис. 3) и рост увеличивается, (Рис. 4) но это можно списать на отсутствие "старых" мечей в первые месяцы.
\end{frame}


\begin{frame}
\frametitle{Анализ и выводы\\{\small График, показывающий качество работы кузнецов (поломанные мечи / произведенные)}}
\begin{figure}[h]
		\includegraphics[width=40mm]{5.png}
		\caption{5}
		\label{Fifth}
\end{figure}

В целом, видно, что кузнецы, ковавшие из стали harpy.co, справились с работой лучше, но так же видно, что более предсказуема работа кузнецов над westeros.inc, т.к. ближе к соотношению (поломанные мечи / произведенные) = const.
\end{frame}


\begin{frame}
\frametitle{Анализ результатов}
harpy.co лучше для коротких войн, т.к. очень стабильны в первые 3 месяца, \\далее - увеличивается риск.\\
~\\
westeros.inc - образец стабильности. Заказывать для длительных походов и не волноваться.
\end{frame}

\begin{frame}{Задание выполняли}
	\begin{itemize}
		{\small
		\item Никита Ларичев, студент 311 группы. Написание кода в ipython notebook.
		\item Наталья Язовская, студентка 312 группы. Оформление презентации средствами latex и beamer.}
	\end{itemize}	

\end{frame}


\end{document}