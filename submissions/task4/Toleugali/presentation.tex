\documentclass[8pt,pdf,hyperref={unicode}]{beamer}
\usepackage{lmodern} 
\usepackage[T2A]{fontenc}
\usepackage[utf8]{inputenc}
\setbeamertemplate{navigation symbols}{}
\usetheme{Warsaw}
\usecolortheme{seahorse}
		\title{Практикум на ЭВМ.\\Отчет о выполнении II задания}
		\institute{Московский Государственный Университет М.В Ломоносова\\
					Факультет Вычислительной математики и Кибернетики\\
						Кафедра Исследования Операций\\}
		\author{Толеугали Мадияр, 312 группа}
		\date{Москва 2017}
\begin{document}
	\begin{frame}
		\titlepage
	\end{frame}
	
	\begin{frame}	
		\frametitle{Постановка задачи}
Для ведения войны с Семью Королевствами нужно оружие, а для оружия нужна сталь. Нет никаких сомнений в кузнечном искусстве Безупречных, однако поставщики стали не столь надежны.
Два основных поставщика стали- Westeros Inc. и Harpy $\&$ Co. На протяжении нескольких месяцев мы закупаем сталь у обеих компаний.  Таким образом, у нас есть данные в формате CSV о производстве оружия и количестве единиц сломанного оружия в каждый из месяцев ведения боевых действий.  
Каждая из компаний предлагает ощутимую скидку при заключении эксклюзивного договора на поставку. С каким поставщиком стали следует заключить договор?
	\end{frame}
	
	\begin{frame}
		\frametitle{ Резюме о проделанной работе }
		\begin{itemize}
		\item Использованы несколько показателей и значений:\\
		\begin{enumerate}
			\item Среднее количество поломок мечей из стали каждой из компаний, их объем производства\\
			\item Срок службы меча. 
			\item Доля поломок в каждом месяце.
			\item Мечи на два месяца.

		\end{enumerate}
		\end{itemize}	
		В результате, по всем показателям сталь фирмы Harpy оказалась лучше. Но были сделаны несколько промежуточных выводов, которые могут немного помочь в выборе стратегии войны, относительно выбора поставщика стали.		
	\end{frame}		
		
	\begin{frame}
	\frametitle{Анализ и выводы}
	\framesubtitle{Среднее количество поломок}
		\begin{itemize}
			\item  Было определено, что при примерно одинаковой загрузке(одинаковом объеме производства), среднее количество поломок мечей, сделанной из стали Westeros.inc, больше, в целом, за все время.
			\includegraphics[scale=0.28]{1.png}
			\includegraphics[scale=0.28]{2.png} 
		\end{itemize}
	\end{frame}
			
	\begin{frame}
		\frametitle{Анализ и выводы}
	\framesubtitle{Срок службы меча.}
		\begin{itemize}
			\item  Доля сломанных мечей от общего числа произведенных, в зависимости от месяца, в который они были произведены. Таким образом, можно заметить, что в краткосрочной перспективе(до 3 месяцев) мечи стали Harpy лучше. Затем можно сделать вывод о том, что в долгосрочной перспективе, все таки, лучше выбрать Westeros- доля сломанных мечей уменьшается стремительно. Но эти выводы сделаны для нескольких месяцев в совокупности(срок служения меча), то есть если какой-то единственный месяц был более удачным или же наоборот очень неудачным, то данный показатель может значительно терять адекватность и точность.\\
			\center{\includegraphics[scale=0.28]{2.png}}
			
		\end{itemize}
	\end{frame}
	
	\begin{frame}
	\frametitle{Анализ и выводы}
	\framesubtitle{Доля поломок в каждом месяце.}
	\begin{itemize}
		\item  Рассмотрев доли сломанных мечей в каждом месяце, мы видим, что изначально мечей, сделанных из стали Westeros, ломается больше и со временем интенсивность поломок уменьшается. В то время, как у Harpy после 3 месяца мечей ломается больше, но на протяжении всего времени доля среднего числа поломок у этой фирмы меньше. \\
		a. Plot \\ 
		b. Boxplot \\
		
		\includegraphics[scale=0.28]{3.png}	
		\includegraphics[scale=0.28]{4.png}		 
	\end{itemize}
	\end{frame}	
	
	\begin{frame}
	\frametitle{Анализ и выводы}
	\framesubtitle{Мечи на два месяца.}
	\begin{itemize}
		\item   Мы знаем, что нам поставляют мечи ежемесячно. Сломанные через долгий срок мечи, мы можем заменить, а категория запроса “здесь и сейчас” – является приоритетом. Поэтому мы рассмотрели показатель, определяющий долю поломок для мечей за 2 месяца их использования. (по 6-ому месяцу нет данных о 8-ом месяце). Таким образом, из графика видно значительное преобладание стали фирмы Harpy. По сути, он подтверждает то, что в какой бы месяц мы не закупили сталь, в среднем, в ближайшее время сломанных мечей, сделанных из стали Harpy, будет почти в 3 раза меньше, что позволит создавать резерв на будущее.  				
		\center{\includegraphics[scale=0.28]{5.png}}
		\end{itemize}
	\end{frame}
	\begin{frame}
	\frametitle{Вывод и результаты}
	Мы увидели, что по всем рассмотренным нами показателям сталь Harpy.co лучше.  Но стоит обратить внимание на следующие вещи:
Мечам из стали Harpy.Co свойственно служить короткий срок, в связи с чем, нам необходимо создавать резерв(учитывать амортизацию).
При выборе стали Westeros есть большой риск проиграть войну сегодня, а завтра мечи и не пригодятся.
	\end{frame}
	
	
\end{document}
